\documentclass[conference]{IEEEtran}
\IEEEoverridecommandlockouts
% The preceding line is only needed to identify funding in the first footnote. If that is unneeded, please comment it out.
\usepackage{cite}
\usepackage{amsmath,amssymb,amsfonts}
\usepackage{algorithmic}
\usepackage{graphicx}
\usepackage{textcomp}
\usepackage{xcolor}
\def\BibTeX{{\rm B\kern-.05em{\sc i\kern-.025em b}\kern-.08em
    T\kern-.1667em\lower.7ex\hbox{E}\kern-.125emX}}
\begin{document}

\title{Evaluating the Impact of Interactivity on User Engagement in Virtual Reality Museums: A Comparative Study}\\

\author{\IEEEauthorblockN{1\textsuperscript{st}I Putu Bagus Gede Prasetyo\\Raharja}
\IEEEauthorblockA{\textit{Department of Informatics} \\
\textit{Institut Teknologi Sepuluh Nopember }\\
Surabaya, Indonesia \\
6025231010@student.its.ac.id}
\and
\IEEEauthorblockN{2\textsuperscript{nd}Muhammad Shafhi Kasyfillah}
\IEEEauthorblockA{\textit{Department of Informatics} \\
\textit{Institut Teknologi Sepuluh Nopember }\\
Surabaya, Indonesia \\
6025231053@student.its.ac.id}
\and
\IEEEauthorblockN{3\textsuperscript{rd}I Nyoman Gde Artadana\\Mahaputra Wardhiana}
\IEEEauthorblockA{\textit{Department of Informatics} \\
\textit{Institut Teknologi Sepuluh Nopember }\\
Surabaya, Indonesia \\
6025231022@student.its.ac.id}
\and
\IEEEauthorblockN{5\textsuperscript{th}Hadziq Fabroyir}
\IEEEauthorblockA{\textit{Department of Informatics} \\
\textit{Institut Teknologi Sepuluh Nopember }\\
Surabaya, Indonesia \\
Hadziq@its.ac.id}
}

\maketitle

\begin{abstract}
This document is a model and instructions for \LaTeX.
This and the IEEEtran.cls file define the components of your paper [title, text, heads, etc.]. *CRITICAL: Do Not Use Symbols, Special Characters, Footnotes, 
or Math in Paper Title or Abstract.
\end{abstract}

\begin{IEEEkeywords}
component, formatting, style, styling, insert
\end{IEEEkeywords}

\section{Introduction}
\IEEEPARstart{T}{he} integration of virtual reality (VR) technology in museums presents a significant opportunity to enhance user engagement, learning outcomes, and overall user experience. Previous research has shown that VR technology can enhance learning effectiveness and user experience by increasing perceived presence, immersion, realism, and satisfaction​. However, these outcomes can vary significantly based on the level of interactivity and the design of the VR interface.

This study aims to explore the effects of different levels of interactivity in VR museum exhibits on these variables, specifically focusing on Balinese traditional masks. Two VR museums will be created: one with static displays and plain sight explanations, and another with enhanced interactivity, including features such as holding the masks virtually and multimedia content like sound explanations.

By investigating these differences, this study aims to provide empirical evidence on the impact of VR interactivity in digital museums, contributing to the development of more effective and engaging VR educational tools. This research will utilize quantitative methods to measure engagement, comprehension, and usability, including time spent on tours, questionnaire scores, and System Usability Scale (SUS) scores. The findings will help in understanding how different VR designs can influence user experiences and learning outcomes, thereby supporting the hypothesis that increased interactivity enhances the effectiveness of VR museum exhibits.

\section{Related Works}
The integration of virtual reality (VR) into educational and museum settings has generated substantial academic interest. This section reviews key studies on the effectiveness of VR technology in enhancing user engagement, learning outcomes, and overall experience, with a particular focus on the role of interactivity. By examining these related works, the current study is contextualized within existing literature, highlighting the gaps it seeks to address.

Rahimi et al.~\cite{9286680} examine the impact of integrating virtual reality (VR) with physical exhibits in museums. Their research shows that VR-enhanced environments significantly boost learning and enjoyment compared to traditional and video-enhanced settings. By experimenting with different exhibit formats, the study provides valuable insights into how VR technology can create engaging and educational museum experiences, offering a promising direction for future hybrid museum spaces.

Kim et al.~\cite{6797425} discuss the development of an interactive VR interface for archaeological research and education. Focusing on the Northwest Palace of King Ashurnasirpal II in Iraq, the project integrates precise archaeological data into a VR environment, offering full-body immersion and user interaction with virtual artifacts. This VR museum aims to preserve and demonstrate cultural heritage, addressing the challenges of on-site conservation and providing a valuable tool for scholars and the public.

Despite these advancements, there remains a need for more empirical research on the specific effects of varying levels of interactivity in VR museum settings. Most existing studies focus broadly on the benefits of VR without isolating the impact of interactivity. This study aims to fill this gap by providing a comparative analysis of static versus interactive VR museum exhibits. By focusing on Balinese traditional masks, this research will offer unique insights into how interactivity influences user engagement, comprehension, and overall user experience in a cultural heritage context.

\section{Methodology}

This study employs a comparative design to investigate the impact of different levels of interactivity in virtual reality (VR) museum exhibits on user engagement, comprehension, and usability. Two VR museum environments featuring Balinese traditional masks will be developed: one with static displays and plain sight explanations, and another with enhanced interactivity, including features such as virtual manipulation of the masks and multimedia content.

\subsection{Participants}
The study will involve Balinese university students majoring in Information Technology, aged 20-22. This demographic is selected due to their likely familiarity with digital technologies and potential exposure to VR environments, making them suitable candidates for this study. Participation will be voluntary, and informed consent will be obtained from all participants.

\subsection{Independent Variable}
The primary independent variable is the type of VR museum exhibit:
\begin{itemize}
\item Non-Interactive VR Museum: This setting presents the masks in a static display format with explanations provided in plain sight, similar to traditional museum exhibits.
\item Interactive VR Museum: This setting enhances user interactivity with features such as virtual manipulation of the masks, zooming for detailed textures, rotating masks, and accessing multimedia descriptions, including audio explanations.
\end{itemize}
\subsection{Controlled Variables}
Several controlled variables will be maintained to ensure consistency:
\begin{itemize}
\item Content Consistency: The cultural and historical information about each mask will be identical in both settings to isolate the impact of interactivity.
\item Navigation Experience: The navigation mechanics will be consistent between the two settings to avoid confounding results with differences in navigation ease or comfort.
\item Participant Demographics: The study will target Balinese university students to control for variations in background knowledge or familiarity with VR technology.
Dependent Variables
\item User Engagement: Measured by the time spent on each museum tour and the scores on the Immersive Tendencies Questionnaire (ITQ), which evaluates baseline immersive tendencies.
Comprehension and Retention: Assessed through post-tour questionnaires that measure participants' understanding and retention of the information presented.
\item Usability: Evaluated using the System Usability Scale (SUS) scores obtained after the tour.
\end{itemize}
\subsection{Apparatus and Materials}
\begin{itemize}
\item VR Headsets: High-quality VR headsets (e.g., Oculus Rift or HTC Vive) will be used to provide an immersive experience.
\item Computers: High-performance computers capable of running VR applications smoothly.
\item VR Controllers: Handheld controllers compatible with the VR headsets to allow interaction with virtual objects.
\item Audio Equipment: High-quality headphones to deliver immersive audio explanations and background sounds.
\item Questionnaires: Digital versions of the ITQ and SUS questionnaires for assessing immersive tendencies and usability.
\item Environment Setup: A quiet room with minimal distractions for participants to use the VR equipment comfortably.
\end{itemize}
\subsection{Preparation}
\begin{itemize}
\item Recruitment: Recruit participants meeting the specified demographic criteria.
\item Briefing: Provide an overview of the study, including its purpose and participation requirements.
\item Consent: Obtain informed consent from all participants.
\item ITQ Administration: Have participants complete the ITQ to assess their baseline immersive tendencies.
\end{itemize}
\subsection{Experiment Sessions}
\begin{itemize}
\item Session Assignment: Randomly assign participants to experience either the non-interactive or interactive VR museum first.
\item Non-Interactive VR Museum: Participants explore the static VR museum with plain sight explanations. Record the time spent on the tour.
\item Interactive VR Museum: Participants interact with the VR museum, utilizing features like virtual manipulation of masks and multimedia content. Record the time spent on the tour.
\item Post-Tour Questionnaire: Administer a questionnaire to assess participants' understanding and retention of information about the masks.
\item SUS Administration: Have participants complete the SUS to evaluate the usability of the VR application.
\item Debriefing Session: Explain the study's purpose and answer any questions from participants.
\item Feedback Collection: Collect feedback on participants' experiences in both VR museum settings.
\item Data Analysis
\item Data Compilation: Compile data from ITQ, time spent, post-tour questionnaires, and SUS scores.
\item Statistical Analysis: Conduct statistical analyses to compare user engagement, comprehension, retention, and usability between the non-interactive and interactive VR museums using the Mann-Whitney U test to determine significant differences.
\end{itemize}
\subsection{Data Analysis}
Data will be analyzed using the Mann-Whitney U test, a non-parametric test suitable for comparing two independent groups when normality assumptions are not met. The steps include:
\begin{itemize}
\item Data Compilation: Gather data from ITQ, time spent on tours, post-tour questionnaires, and SUS scores for both VR museum settings.
Descriptive Statistics: Calculate the median and interquartile range (IQR) for each dependent variable in both groups.
\item Mann-Whitney U Test: Perform the test for each dependent variable, comparing the U statistic to critical values and determining p-values.
\item Interpretation of Results: Assess significance by comparing p-values to the 0.05 threshold, rejecting the null hypothesis if p-values are below this level.
\item Reporting: Report median, IQR, U statistic, and p-values for each dependent variable, detailing significant findings.
\end{itemize}

\section{Prepare Your Paper Before Styling}
Before you begin to format your paper, first write and save the content as a 
separate text file. Complete all content and organizational editing before 
formatting. Please note sections \ref{AA}--\ref{SCM} below for more information on 
proofreading, spelling and grammar.

Keep your text and graphic files separate until after the text has been 
formatted and styled. Do not number text heads---{\LaTeX} will do that 
for you.

\subsection{Abbreviations and Acronyms}\label{AA}
Define abbreviations and acronyms the first time they are used in the text, 
even after they have been defined in the abstract. Abbreviations such as 
IEEE, SI, MKS, CGS, ac, dc, and rms do not have to be defined. Do not use 
abbreviations in the title or heads unless they are unavoidable.

\subsection{Units}
\begin{itemize}
\item Use either SI (MKS) or CGS as primary units. (SI units are encouraged.) English units may be used as secondary units (in parentheses). An exception would be the use of English units as identifiers in trade, such as ``3.5-inch disk drive''.
\item Avoid combining SI and CGS units, such as current in amperes and magnetic field in oersteds. This often leads to confusion because equations do not balance dimensionally. If you must use mixed units, clearly state the units for each quantity that you use in an equation.
\item Do not mix complete spellings and abbreviations of units: ``Wb/m\textsuperscript{2}'' or ``webers per square meter'', not ``webers/m\textsuperscript{2}''. Spell out units when they appear in text: ``. . . a few henries'', not ``. . . a few H''.
\item Use a zero before decimal points: ``0.25'', not ``.25''. Use ``cm\textsuperscript{3}'', not ``cc''.)
\end{itemize}

\subsection{Equations}
Number equations consecutively. To make your 
equations more compact, you may use the solidus (~/~), the exp function, or 
appropriate exponents. Italicize Roman symbols for quantities and variables, 
but not Greek symbols. Use a long dash rather than a hyphen for a minus 
sign. Punctuate equations with commas or periods when they are part of a 
sentence, as in:
\begin{equation}
a+b=\gamma\label{eq}
\end{equation}

Be sure that the 
symbols in your equation have been defined before or immediately following 
the equation. Use ``\eqref{eq}'', not ``Eq.~\eqref{eq}'' or ``equation \eqref{eq}'', except at 
the beginning of a sentence: ``Equation \eqref{eq} is . . .''

\subsection{\LaTeX-Specific Advice}

Please use ``soft'' (e.g., \verb|\eqref{Eq}|) cross references instead
of ``hard'' references (e.g., \verb|(1)|). That will make it possible
to combine sections, add equations, or change the order of figures or
citations without having to go through the file line by line.

Please don't use the \verb|{eqnarray}| equation environment. Use
\verb|{align}| or \verb|{IEEEeqnarray}| instead. The \verb|{eqnarray}|
environment leaves unsightly spaces around relation symbols.

Please note that the \verb|{subequations}| environment in {\LaTeX}
will increment the main equation counter even when there are no
equation numbers displayed. If you forget that, you might write an
article in which the equation numbers skip from (17) to (20), causing
the copy editors to wonder if you've discovered a new method of
counting.

{\BibTeX} does not work by magic. It doesn't get the bibliographic
data from thin air but from .bib files. If you use {\BibTeX} to produce a
bibliography you must send the .bib files. 

{\LaTeX} can't read your mind. If you assign the same label to a
subsubsection and a table, you might find that Table I has been cross
referenced as Table IV-B3. 

{\LaTeX} does not have precognitive abilities. If you put a
\verb|\label| command before the command that updates the counter it's
supposed to be using, the label will pick up the last counter to be
cross referenced instead. In particular, a \verb|\label| command
should not go before the caption of a figure or a table.

Do not use \verb|\nonumber| inside the \verb|{array}| environment. It
will not stop equation numbers inside \verb|{array}| (there won't be
any anyway) and it might stop a wanted equation number in the
surrounding equation.

\subsection{Some Common Mistakes}\label{SCM}
\begin{itemize}
\item The word ``data'' is plural, not singular.
\item The subscript for the permeability of vacuum $\mu_{0}$, and other common scientific constants, is zero with subscript formatting, not a lowercase letter ``o''.
\item In American English, commas, semicolons, periods, question and exclamation marks are located within quotation marks only when a complete thought or name is cited, such as a title or full quotation. When quotation marks are used, instead of a bold or italic typeface, to highlight a word or phrase, punctuation should appear outside of the quotation marks. A parenthetical phrase or statement at the end of a sentence is punctuated outside of the closing parenthesis (like this). (A parenthetical sentence is punctuated within the parentheses.)
\item A graph within a graph is an ``inset'', not an ``insert''. The word alternatively is preferred to the word ``alternately'' (unless you really mean something that alternates).
\item Do not use the word ``essentially'' to mean ``approximately'' or ``effectively''.
\item In your paper title, if the words ``that uses'' can accurately replace the word ``using'', capitalize the ``u''; if not, keep using lower-cased.
\item Be aware of the different meanings of the homophones ``affect'' and ``effect'', ``complement'' and ``compliment'', ``discreet'' and ``discrete'', ``principal'' and ``principle''.
\item Do not confuse ``imply'' and ``infer''.
\item The prefix ``non'' is not a word; it should be joined to the word it modifies, usually without a hyphen.
\item There is no period after the ``et'' in the Latin abbreviation ``et al.''.
\item The abbreviation ``i.e.'' means ``that is'', and the abbreviation ``e.g.'' means ``for example''.
\end{itemize}
An excellent style manual for science writers is \cite{b7}.

\subsection{Authors and Affiliations}
\textbf{The class file is designed for, but not limited to, six authors.} A 
minimum of one author is required for all conference articles. Author names 
should be listed starting from left to right and then moving down to the 
next line. This is the author sequence that will be used in future citations 
and by indexing services. Names should not be listed in columns nor group by 
affiliation. Please keep your affiliations as succinct as possible (for 
example, do not differentiate among departments of the same organization).

\subsection{Identify the Headings}
Headings, or heads, are organizational devices that guide the reader through 
your paper. There are two types: component heads and text heads.

Component heads identify the different components of your paper and are not 
topically subordinate to each other. Examples include Acknowledgments and 
References and, for these, the correct style to use is ``Heading 5''. Use 
``figure caption'' for your Figure captions, and ``table head'' for your 
table title. Run-in heads, such as ``Abstract'', will require you to apply a 
style (in this case, italic) in addition to the style provided by the drop 
down menu to differentiate the head from the text.

Text heads organize the topics on a relational, hierarchical basis. For 
example, the paper title is the primary text head because all subsequent 
material relates and elaborates on this one topic. If there are two or more 
sub-topics, the next level head (uppercase Roman numerals) should be used 
and, conversely, if there are not at least two sub-topics, then no subheads 
should be introduced.

\subsection{Figures and Tables}
\paragraph{Positioning Figures and Tables} Place figures and tables at the top and 
bottom of columns. Avoid placing them in the middle of columns. Large 
figures and tables may span across both columns. Figure captions should be 
below the figures; table heads should appear above the tables. Insert 
figures and tables after they are cited in the text. Use the abbreviation 
``Fig.~\ref{fig}'', even at the beginning of a sentence.

\begin{table}[htbp]
\caption{Table Type Styles}
\begin{center}
\begin{tabular}{|c|c|c|c|}
\hline
\textbf{Table}&\multicolumn{3}{|c|}{\textbf{Table Column Head}} \\
\cline{2-4} 
\textbf{Head} & \textbf{\textit{Table column subhead}}& \textbf{\textit{Subhead}}& \textbf{\textit{Subhead}} \\
\hline
copy& More table copy$^{\mathrm{a}}$& &  \\
\hline
\multicolumn{4}{l}{$^{\mathrm{a}}$Sample of a Table footnote.}
\end{tabular}
\label{tab1}
\end{center}
\end{table}

\begin{figure}[htbp]
\centerline{\includegraphics{fig1.png}}
\caption{Example of a figure caption.}
\label{fig}
\end{figure}

Figure Labels: Use 8 point Times New Roman for Figure labels. Use words 
rather than symbols or abbreviations when writing Figure axis labels to 
avoid confusing the reader. As an example, write the quantity 
``Magnetization'', or ``Magnetization, M'', not just ``M''. If including 
units in the label, present them within parentheses. Do not label axes only 
with units. In the example, write ``Magnetization (A/m)'' or ``Magnetization 
\{A[m(1)]\}'', not just ``A/m''. Do not label axes with a ratio of 
quantities and units. For example, write ``Temperature (K)'', not 
``Temperature/K''.

\section*{Acknowledgment}

The preferred spelling of the word ``acknowledgment'' in America is without 
an ``e'' after the ``g''. Avoid the stilted expression ``one of us (R. B. 
G.) thanks $\ldots$''. Instead, try ``R. B. G. thanks$\ldots$''. Put sponsor 
acknowledgments in the unnumbered footnote on the first page.

\bibliographystyle{IEEEtran}
\bibliography{references}


\end{document}
